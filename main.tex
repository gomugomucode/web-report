\documentclass[a4paper,12pt]{article}

% Setting up the preamble
\usepackage[utf8]{inputenc}
\usepackage[T1]{fontenc}
\usepackage{geometry}
\geometry{margin=1in}
\usepackage{listings}
\usepackage{xcolor}
\usepackage{fancyhdr}
\usepackage{graphicx}
\usepackage{float}
\usepackage{tocloft}
\usepackage{parskip}
\usepackage{noto}

% Configuring listings for code display
\lstset{
    basicstyle=\ttfamily\small,
    breaklines=true,
    frame=single,
    numbers=none,
    keywordstyle=\color{blue},
    commentstyle=\color{gray},
    stringstyle=\color{red},
    showstringspaces=false,
    tabsize=2,
}

% Configure header and footer
\pagestyle{fancy}
\fancyhf{} % Clear header and footer
\renewcommand{\headrulewidth}{0pt} % Remove header rule
\cfoot{\thepage} % Optional: Add page number to footer

% Remove date
\date{} % Empty date to remove it
\title{Project Report} % Optional: Define title if needed
\author{Your Name} % Optional: Define author if needed

\begin{document}

% Center the Table of Contents title and add 2-3 line gap
{\centering \renewcommand{\contentsname}{ Table of Contents } \par}
\vspace{3\baselineskip} 
\tableofcontents
\newpage

% Include your technologies.tex file
% technologies.tex
\section{Technologies Used}
This section describes the key technologies used in the student registration system, outlining their purpose, application, and foundational structures (boilerplates and styling methods).

\begin{description}
    \item[HTML (HyperText Markup Language)] HTML is the standard markup language for creating the structure of web pages. It defines elements like forms, tables, and navigation bars using tags (e.g., \texttt{<form>}, \texttt{<table>}, \texttt{<nav>}). In this project, HTML is used to create the student registration form, student information table, and website structure with a navbar, landing page, and footer. Semantic tags (e.g., \texttt{<header>}, \texttt{<nav>}) and form validation attributes (e.g., \texttt{required}) ensure accessibility and functionality.

    \subsection{HTML}
    The HTML boilerplate is the standard structure for an HTML document, providing essential tags and metadata for web compatibility. Below is a basic HTML5 boilerplate used as the foundation for project files like the student registration form and website.

    \lstset{language=HTML}
    \begin{lstlisting}
<!DOCTYPE html>
<html lang="en">
<head>
    <meta charset="UTF-8">
    <meta name="viewport" content="width=device-width, initial-scale=1.0">
    <title>Document Title</title>
</head>
<body>
    <!-- Content goes here -->
</body>
</html>
    \end{lstlisting}
    Key elements include:
    \begin{itemize}
        \item \texttt{<!DOCTYPE html>}: Declares the document as HTML5.
        \item \texttt{<html lang="en">}: Specifies the document language.
        \item \texttt{<meta charset="UTF-8">}: Ensures proper character encoding.
        \item \texttt{<meta name="viewport">}: Enables responsive design for mobile devices.
        \item \texttt{<title>}: Sets the page title displayed in the browser.
    \end{itemize}

    \item[CSS (Cascading Style Sheets)] CSS styles and layouts HTML elements, controlling visual aspects like colors, fonts, and responsiveness. In this project, CSS enhances the registration form with gradient backgrounds and flexbox layouts, styles the student information table with borders and shadows, and designs the website with a fixed navbar and responsive media queries. Transitions and hover effects (e.g., \texttt{.btn:hover}) improve user interaction.

    \subsection{CSS Inclusion Methods}
    CSS can be included in HTML in three ways: inline, internal, and external. Each method is demonstrated below.

    \subsubsection{Inline CSS}
    Inline CSS applies styles directly to an HTML element using the \texttt{style} attribute. It is used for quick, element-specific styling but is less maintainable for large projects.
    \lstset{language=HTML}
    \begin{lstlisting}
<div style="background-color: #f0f0f0; padding: 10px;">Inline styled div</div>
    \end{lstlisting}

    \subsubsection{Internal CSS}
    Internal CSS is defined within a \texttt{<style>} tag in the HTML \texttt{<head>}. It is suitable for single-page styling, as used in the student registration form and website in this project.
    \lstset{language=HTML}
    \begin{lstlisting}
<head>
    <style>
        .container {
            background-color: #f0f0f0;
            padding: 10px;
        }
    </style>
</head>
<body>
    <div class="container">Internal styled div</div>
</body>
    \end{lstlisting}

    \subsubsection{External CSS}
    External CSS is stored in a separate \texttt{.css} file and linked using the \texttt{<link>} tag. It is ideal for applying consistent styles across multiple pages, promoting maintainability.
    \lstset{language=HTML}
    \begin{lstlisting}
<head>
    <link rel="stylesheet" href="styles.css">
</head>
<body>
    <div class="container">External styled div</div>
</body>
    \end{lstlisting}
    \lstset{language=CSS}
    \begin{lstlisting}
/* styles.css */
.container {
    background-color: #f0f0f0;
    padding: 10px;
}
    \end{lstlisting}

    \item[PHP (Hypertext Preprocessor)] PHP is a server-side scripting language for dynamic web content and database interaction. It processes form submissions and performs CRUD operations (Create, Read, Update, Delete). In this project, PHP scripts (e.g., \texttt{insert.php}, \texttt{display.php}) manage student data in a MySQL database using prepared statements for security and generate dynamic HTML for data display.

    \subsection{PHP }
    The PHP boilerplate provides a basic structure for a PHP script interacting with a MySQL database, as used in the project’s CRUD operations. It includes database connection setup and error handling.
    \lstset{language=PHP}
    \begin{lstlisting}
<?php
$servername = "localhost";
$username = "root";
$password = "";
$database = "database_name";
$conn = new mysqli($servername, $username, $password, $database);
if ($conn->connect_error) {
    die("Connection failed: " . $conn->connect_error);
}
// Perform database operations here
$conn->close();
?>
    \end{lstlisting}
    Key components include:
    \begin{itemize}
        \item \texttt{<?php ... ?>}: Delimits PHP code.
        \item \texttt{mysqli}: Establishes a MySQL connection.
        \item Error handling with \texttt{die()}: Terminates execution if the connection fails.
        \item \texttt{\$conn->close()}: Closes the database connection to free resources.
    \end{itemize}

    \item[XML (eXtensible Markup Language)] XML stores and transports structured data with custom tags and schema validation (e.g., XSD). In this project, XML structures student registration data (\texttt{<studentRegistration>}) and employee records (\texttt{<employeeRecords>}), with an XSD schema validating employee data for attributes like phone number format and age range. It ensures data consistency and portability.

    \subsection{XML }
    The XML boilerplate defines the basic structure of an XML document, including the XML declaration and a root element. It may reference an XSD schema for validation, as used in the employee records.
    \lstset{language=XML}
    \begin{lstlisting}
<?xml version="1.0" encoding="UTF-8"?>
<root>
    <!-- Data elements go here -->
</root>
    \end{lstlisting}
    With an XSD schema reference:
    \lstset{language=XML}
    \begin{lstlisting}
<?xml version="1.0" encoding="UTF-8"?>
<root xmlns:xsi="http://www.w3.org/2001/XMLSchema-instance"
      xsi:noNamespaceSchemaLocation="schema.xsd">
    <!-- Data elements go here -->
</root>
    \end{lstlisting}
    Key components include:
    \begin{itemize}
        \item \texttt{<?xml version="1.0" encoding="UTF-8"?>}: Specifies the XML version and encoding.
        \item \texttt{<root>}: The mandatory root element containing all data.
        \item \texttt{xsi:noNamespaceSchemaLocation}: Links to an XSD schema for validation (optional).
    \end{itemize}
\end{description}

\clearpage

\section{Practical}
This lab report implements a student registration system using HTML, CSS, PHP, and MySQL, with XML for structured data. It includes a registration form, a student information table, CRUD operations, and XML examples for student and employee data. Output images are included to demonstrate the rendered results. All code listings are formatted consistently with a monospaced font, small size, and single-line spacing.

\section{Student Registration Form}
The form collects roll number, personal details, faculty, subjects, photo, and comments, styled with CSS and validated with JavaScript.

\lstset{language=HTML}
\begin{lstlisting}
<!DOCTYPE html>
<html lang="en">
<head>
  <title>Student Registration Form</title>
  <meta charset="UTF-8">
  <meta name="viewport" content="width=device-width, initial-scale=1.0">
  <style>
    * {
      margin: 0;
      padding: 0;
      box-sizing: border-box;
    }
    body {
      font-family: 'Segoe UI', Tahoma, Geneva, Verdana, sans-serif;
      background: linear-gradient(135deg, #667eea 0%, #764ba2 100%);
      min-height: 100vh;
      padding: 20px;
    }
    .container {
      max-width: 800px;
      margin: 0 auto;
      background: white;
      border-radius: 15px;
      box-shadow: 0 20px 40px rgba(0, 0, 0, 0.1);
      overflow: hidden;
    }
    .header {
      background: linear-gradient(135deg, #4facfe 0%, #00f2fe 100%);
      color: white;
      text-align: center;
      padding: 30px;
    }
    .header h1 {
      font-size: 2rem;
      margin-bottom: 10px;
    }
    .form-container {
      padding: 40px;
    }
    .form-row {
      display: flex;
      margin-bottom: 25px;
      align-items: flex-start;
      gap: 20px;
    }
    .form-row label {
      flex: 0 0 200px;
      font-weight: 600;
      color: #333;
      padding-top: 8px;
    }
    input[type="text"],
    input[type="number"],
    input[type="tel"],
    input[type="date"],
    input[type="file"],
    select,
    textarea {
      width: 100%;
      padding: 12px 15px;
      border: 2px solid #e1e5e9;
      border-radius: 8px;
      font-size: 14px;
      transition: all 0.3s ease;
      font-family: inherit;
    }
    .radio-item,
    .checkbox-item {
      display: flex;
      align-items: center;
      gap: 8px;
    }
    input[type="radio"],
    input[type="checkbox"] {
      width: auto;
      margin: 0;
      transform: scale(1.2);
    }
    textarea {
      resize: vertical;
      min-height: 100px;
    }
    .button-group {
      display: flex;
      gap: 15px;
      justify-content: center;
      margin-top: 30px;
    }
    .btn {
      padding: 12px 30px;
      border: none;
      border-radius: 8px;
      font-size: 16px;
      font-weight: 600;
      cursor: pointer;
      transition: all 0.3s ease;
      text-transform: uppercase;
      letter-spacing: 0.5px;
    }
    .btn-primary {
      background-color: #4facfe;
      color: white;
    }
    .btn-secondary {
      background-color: #ccc;
      color: #333;
    }
  </style>
</head>
<body>
  <div class="container">
    <div class="header">
      <h1>Student Registration</h1>
      <p>Please fill out all required information</p>
    </div>
    <div class="form-container">
      <form id="studentForm">
        <div class="form-row">
          <label for="rollNo">Roll Number</label>
          <div class="input-group">
            <input type="number" id="rollNo" name="rollNo" placeholder="Enter your roll number" required>
          </div>
        </div>
        <div class="form-row">
          <label for="fullName">Full Name</label>
          <div class="input-group">
            <input type="text" id="fullName" name="fullName" placeholder="Enter student's full name" required>
          </div>
        </div>
        <div class="form-row">
          <label for="address">Address</label>
          <div class="input-group">
            <input type="text" id="address" name="address" placeholder="Enter student's address" required>
          </div>
        </div>
        <div class="form-row">
          <label for="phone">Phone Number</label>
          <div class="input-group">
            <input type="tel" id="phone" name="phone" placeholder="Enter phone number" required>
          </div>
        </div>
        <div class="form-row">
          <label>Gender</label>
          <div class="input-group">
            <div class="radio-group">
              <div class="radio-item">
                <input type="radio" id="male" name="gender" value="male" required>
                <label for="male">Male</label>
              </div>
              <div class="radio-item">
                <input type="radio" id="female" name="gender" value="female" required>
                <label for="female">Female</label>
              </div>
              <div class="radio-item">
                <input type="radio" id="other" name="gender" value="other" required>
                <label for="other">Other</label>
              </div>
            </div>
          </div>
        </div>
        <div class="form-row">
          <label for="dob">Date of Birth</label>
          <div class="input-group">
            <input type="date" id="dob" name="dob" required>
          </div>
        </div>
        <div class="form-row">
          <label>Subjects of Interest</label>
          <div class="input-group">
            <div class="checkbox-group">
              <div class="checkbox-item">
                <input type="checkbox" id="ai" name="subjects" value="ai">
                <label for="ai">Artificial Intelligence</label>
              </div>
              <div class="checkbox-item">
                <input type="checkbox" id="dba" name="subjects" value="dba">
                <label for="dba">Database Administration</label>
              </div>
              <div class="checkbox-item">
                <input type="checkbox" id="nsa" name="subjects" value="nsa">
                <label for="nsa">Network Security</label>
              </div>
            </div>
          </div>
        </div>
        <div class="form-row">
          <label for="faculty">Select Faculty</label>
          <div class="input-group">
            <select id="faculty" name="faculty" required>
              <option value="">-- Select Faculty --</option>
              <option value="bca">Bachelor of Computer Application (BCA)</option>
              <option value="bba">Bachelor of Business Administration (BBA)</option>
              <option value="bbs">Bachelor of Business Studies (BBS)</option>
            </select>
          </div>
        </div>
        <div class="form-row">
          <label for="photo">Upload Photo</label>
          <div class="input-group">
            <input type="file" id="photo" name="photo" accept="image/*">
          </div>
        </div>
        <div class="form-row full-width">
          <label for="comments">Additional Comments</label>
          <div class="input-group">
            <textarea id="comments" name="comments" placeholder="Write any additional information or comments here..."></textarea>
          </div>
        </div>
        <div class="button-group">
          <button type="submit" class="btn btn-primary">Submit Application</button>
          <button type="reset" class="btn btn-secondary">Reset Form</button>
        </div>
      </form>
    </div>
  </div>
  <script>
    document.getElementById('studentForm').addEventListener('submit', function(e) {
      e.preventDefault();
      alert('Form submitted successfully! (This is a demo)');
    });
    document.querySelector('.btn-secondary').addEventListener('click', function() {
      if (confirm('Are you sure you want to reset the form?')) {
        document.getElementById('studentForm').reset();
      }
    });
  </script>
</body>
</html>
\end{lstlisting}

\subsection{Form Output}
The rendered output of the student registration form is shown below.

\begin{figure}[h]
    \centering
    \includegraphics[width=0.8\textwidth]{10_screenshot.png}
    \caption{Student Registration Form Output}
\end{figure}

\section{Student Information Table}
The table displays student data with CSS styling for a clean look.

\lstset{language=HTML}
\begin{lstlisting}
<!DOCTYPE html>
<html lang="en">
<head>
  <meta charset="UTF-8">
  <meta name="viewport" content="width=device-width, initial-scale=1.0">
  <title>Student Information Table</title>
  <style>
    body {
      font-family: Arial, sans-serif;
      margin: 20px;
      background-color: #f5f5f5;
    }
    .container {
      max-width: 1000px;
      margin: 0 auto;
      background-color: white;
      padding: 30px;
      border-radius: 8px;
      box-shadow: 0 2px 10px rgba(0, 0, 0, 0.1);
    }
    h1 {
      text-align: center;
      color: #333;
      margin-bottom: 30px;
    }
    table {
      width: 100%;
      border-collapse: collapse;
      margin: 20px 0;
      font-size: 16px;
    }
    th {
      background-color: #4CAF50;
      color: white;
      padding: 12px 15px;
      text-align: left;
      font-weight: bold;
    }
    td {
      padding: 12px 15px;
      text-align: left;
      border-bottom: 1px solid #ddd;
    }
    .table-info {
      text-align: center;
      margin-bottom: 20px;
      color: #666;
      font-style: italic;
    }
  </style>
</head>
<body>
  <div class="container">
    <h1>Student Information Database</h1>
    <p class="table-info">Complete list of registered students for Academic Year 2024-25</p>
    <table>
      <thead>
        <tr>
          <th>Roll No</th>
          <th>Student Name</th>
          <th>Age</th>
          <th>Gender</th>
          <th>Faculty</th>
          <th>Email</th>
          <th>Phone Number</th>
          <th>City</th>
        </tr>
      </thead>
      <tbody>
        <tr>
          <td>101</td>
          <td>Rajesh Kumar Sharma</td>
          <td>20</td>
          <td>Male</td>
          <td>BCA</td>
          <td>rajesh.sharma@email.com</td>
          <td>9841234567</td>
          <td>Kathmandu</td>
        </tr>
        <tr>
          <td>102</td>
          <td>Priya Thapa</td>
          <td>19</td>
          <td>Female</td>
          <td>BBA</td>
          <td>priya.thapa@email.com</td>
          <td>9812345678</td>
          <td>Pokhara</td>
        </tr>
        <tr>
          <td>103</td>
          <td>Amit Gurung</td>
          <td>21</td>
          <td>Male</td>
          <td>BBS</td>
          <td>amit.gurung@email.com</td>
          <td>9823456789</td>
          <td>Lalitpur</td>
        </tr>
        <tr>
          <td>104</td>
          <td>Sunita Rai</td>
          <td>20</td>
          <td>Female</td>
          <td>BCA</td>
          <td>sunita.rai@email.com</td>
          <td>9834567890</td>
          <td>Chitwan</td>
        </tr>
      </tbody>
    </table>
    <p class="total-students">Total Students: 4</p>
  </div>
</body>
</html>
\end{lstlisting}

\subsection{Table Output}
The rendered output of the student information table is shown below.

\begin{figure}[h]
    \centering
    \includegraphics[width=0.8\textwidth]{9_screenshot.png}
    \caption{Student Information Table Output}
\end{figure}

\section{Design of website having a navbar, landing page and footer and styling with using necessary properties of html and css like lists(ul, ol , li), effects(hover )}

\subsection{HTML Structure with Internal  CSS}

\begin{lstlisting}[style=htmlstyle, caption=Complete Website Code]
<!DOCTYPE html>
<html lang="en">
<head>
    <meta charset="UTF-8">
    <meta name="viewport" content="width=device-width, initial-scale=1.0">
    <title>My Website</title>
    <style>
        * {
            margin: 0;
            padding: 0;
            box-sizing: border-box;
        }

        body {
            font-family: Arial, sans-serif;
            line-height: 1.6;
        }

        /* Navbar */
        .navbar {
            background-color: #333;
            padding: 1rem 2rem;
            position: fixed;
            top: 0;
            width: 100%;
            z-index: 1000;
        }

        .nav-container {
            display: flex;
            justify-content: space-between;
            align-items: center;
            max-width: 1200px;
            margin: 0 auto;
        }

        .logo {
            color: white;
            font-size: 1.5rem;
            font-weight: bold;
            text-decoration: none;
        }

        .nav-links {
            display: flex;
            list-style: none;
            gap: 2rem;
        }

        .nav-links a {
            color: white;
            text-decoration: none;
            padding: 0.5rem 1rem;
        }

        .nav-links a:hover {
            background-color: #555;
            border-radius: 4px;
        }

        /* Landing Page */
        .landing {
            padding-top: 80px;
            height: calc(100vh - 80px);
            display: flex;
            align-items: center;
            justify-content: center;
            text-align: center;
            background: linear-gradient(135deg, #667eea 0%, #764ba2 100%);
            color: white;
        }

        .landing-content h1 {
            font-size: 3rem;
            margin-bottom: 1rem;
        }

        .landing-content p {
            font-size: 1.2rem;
            margin-bottom: 2rem;
            max-width: 600px;
        }

        .btn {
            background-color: #ff6b6b;
            color: white;
            padding: 12px 30px;
            text-decoration: none;
            border-radius: 5px;
            font-size: 1.1rem;
            display: inline-block;
        }

        .btn:hover {
            background-color: #ff5252;
        }

        /* Footer */
        .footer {
            background-color: #333;
            color: white;
            text-align: center;
            padding: 1rem;
            height: 80px;
            display: flex;
            flex-direction: column;
            justify-content: center;
        }

        .footer p {
            margin-bottom: 0.5rem;
        }

        /* Responsive */
        @media (max-width: 768px) {
            .nav-container {
                flex-direction: column;
                gap: 1rem;
            }

            .nav-links {
                gap: 1rem;
            }

            .landing-content h1 {
                font-size: 2rem;
            }

            .landing-content p {
                font-size: 1rem;
            }
        }
    </style>
</head>
<body>
    <!-- Navbar -->
    <nav class="navbar">
        <div class="nav-container">
            <a href="#" class="logo">MyWebsite</a>
            <ul class="nav-links">
                <li><a href="#home">Home</a></li>
                <li><a href="#about">About</a></li>
                <li><a href="#services">Services</a></li>
                <li><a href="#contact">Contact</a></li>
            </ul>
        </div>
    </nav>

    <!-- Landing Page -->
    <main class="landing" id="home">
        <div class="landing-content">
            <h1>Welcome to My Website</h1>
            <p>This is a simple, clean landing page with a beautiful gradient background. We create amazing experiences and provide excellent services for our customers.</p>
            <a href="#about" class="btn">Get Started</a>
        </div>
    </main>

    <!-- Footer -->
    <footer class="footer">
        <p>&copy; 2025 MyWebsite. All rights reserved.</p>
        <p>Email: info@mywebsite.com | Phone: (555) 123-4567</p>
    </footer>
</body>
</html>
\end{lstlisting}


\subsection{Table Output}
The rendered output of the website is shown below.

\begin{figure}[h]
    \centering
    \includegraphics[width=0.8\textwidth]{13_screenshot.png}
    \caption{My website}
\end{figure}


\section{XML Examples}
This section includes XML structures for student registration and employee records, with an XSD for the employee XML.

\subsection{Student Registration XML}
Corrected XML for student registration data.

\lstset{language=XML}
\begin{lstlisting}
<?xml version="1.0" encoding="UTF-8"?>
<studentRegistration>
  <student>
    <rollNumber>101</rollNumber>
    <fullName>Rajesh Kumar Sharma</fullName>
    <address>123 Main St, Kathmandu</address>
    <phoneNumber>9841234567</phoneNumber>
    <gender>Male</gender>
    <dateOfBirth>2004-05-15</dateOfBirth>
    <subjects>
      <subject>Artificial Intelligence</subject>
      <subject>Database Administration</subject>
    </subjects>
    <faculty>BCA</faculty>
    <comments>Interested in AI research</comments>
  </student>
</studentRegistration>
\end{lstlisting}

\subsection{Employee Records XML with validations}
XML for employee information with its schema.

\lstset{language=XML}
\begin{lstlisting}
<?xml version="1.0" encoding="UTF-8"?>
<employeeRecords>
    <employee>
        <name>Rajesh Kumar Sharma</name>
        <address>
            <temporary>123 Main St, Kathmandu</temporary>
            <permanent>456 Park Ave, Pokhara</permanent>
        </address>
        <phoneNumber>9841234567</phoneNumber>
        <age>30</age>
        <email>rajesh.sharma@company.com</email>
        <workLocation>Kathmandu Branch</workLocation>
    </employee>
</employeeRecords>
\end{lstlisting}

\lstset{language=XML}
\begin{lstlisting}
<?xml version="1.0" encoding="UTF-8"?>
<xs:schema xmlns:xs="http://www.w3.org/2001/XMLSchema">
    <xs:element name="employeeRecords">
        <xs:complexType>
            <xs:sequence>
                <xs:element name="employee" maxOccurs="unbounded">
                    <xs:complexType>
                        <xs:sequence>
                            <xs:element name="name" type="xs:string"/>
                            <xs:element name="address">
                                <xs:complexType>
                                    <xs:sequence>
                                        <xs:element name="temporary" type="xs:string"/>
                                        <xs:element name="permanent" type="xs:string"/>
                                    </xs:sequence>
                                </xs:complexType>
                            </xs:element>
                            <xs:element name="phoneNumber">
                                <xs:simpleType>
                                    <xs:restriction base="xs:string">
                                        <xs:pattern value="[0-9]{10}"/>
                                    </xs:restriction>
                                </xs:simpleType>
                            </xs:element>
                            <xs:element name="age">
                                <xs:simpleType>
                                    <xs:restriction base="xs:positiveInteger">
                                        <xs:minInclusive value="18"/>
                                        <xs:maxInclusive value="65"/>
                                    </xs:restriction>
                                </xs:simpleType>
                            </xs:element>
                            <xs:element name="email">
                                <xs:simpleType>
                                    <xs:restriction base="xs:string">
                                        <xs:pattern value="[a-zA-Z0-9._%+-]+@[a-zA-Z0-9.-]+\.[a-zA-Z]{2,}"/>
                                    </xs:restriction>
                                </xs:simpleType>
                            </xs:element>
                            <xs:element name="workLocation" type="xs:string"/>
                        </xs:sequence>
                    </xs:complexType>
                </xs:element>
            </xs:sequence>
        </xs:complexType>
    </xs:element>
</xs:schema>
\end{lstlisting}

\section{CRUD Operations with PHP}
The CRUD operations manage student data in a MySQL database "STUDENT" with table "BCA" (rollno, name, gender, phone). Each operation is presented with its complete code.

\subsection{Insertion Code}
The insertion operation uses an HTML form (\texttt{form.php}) and a PHP script (\texttt{insert.php}) with prepared statements for security.

\lstset{language=HTML}
\begin{lstlisting}
<!DOCTYPE html>
<html lang="en">
<head>
    <meta charset="UTF-8">
    <title>CRUD Operation</title>
    <style>
        * {
            box-sizing: border-box;
        }
        body {
            margin: 0;
            padding: 0;
            background: linear-gradient(to right, #74ebd5, #ACB6E5);
            font-family: 'Segoe UI', Tahoma, Geneva, Verdana, sans-serif;
            height: 100vh;
            display: flex;
            justify-content: center;
            align-items: center;
        }
        .form-container {
            background: #ffffff;
            padding: 30px 40px;
            border-radius: 15px;
            box-shadow: 0 10px 25px rgba(0, 0, 0, 0.15);
            width: 100%;
            max-width: 400px;
        }
        .form-container h2 {
            text-align: center;
            color: #333;
            margin-bottom: 20px;
        }
        label {
            font-weight: bold;
            margin-bottom: 5px;
            display: block;
            color: #333;
        }
        input[type="text"],
        input[type="number"],
        select {
            width: 100%;
            padding: 10px;
            margin-bottom: 20px;
            border: 2px solid #ccc;
            border-radius: 8px;
            transition: border 0.3s;
        }
        input[type="submit"] {
            width: 100%;
            padding: 12px;
            background-color: #4CAF50;
            border: none;
            border-radius: 8px;
            color: white;
            font-size: 16px;
            font-weight: bold;
            cursor: pointer;
            transition: background 0.3s;
        }
        .message {
            text-align: center;
            margin-bottom: 20px;
            font-weight: bold;
            color: green;
        }
    </style>
</head>
<body>
<div class="form-container">
    <h2>CRUD Operation</h2>
    <?php if (!empty($message)): ?>
        <div class="message <?= strpos($message, 'Error') !== false || strpos($message, 'required') !== false ? 'error' : '' ?>">
            <?= $message ?>
        </div>
    <?php endif; ?>
    <form method="post" action="./insert.php">
        <label for="rollno">Roll No:</label>
        <input type="number" name="rollno" id="rollno" required>
        <label for="name">Name:</label>
        <input type="text" name="name" id="name" required>
        <label for="gender">Gender:</label>
        <select name="gender" id="gender" required>
            <option value="">-- Select Gender --</option>
            <option value="Male">Male</option>
            <option value="Female">Female</option>
            <option value="Other">Other</option>
        </select>
        <label for="phone">Phone:</label>
        <input type="text" name="phone" id="phone" required>
        <input type="submit" value="Submit">
    </form>
    <a href="display.php" style="display: block; text-align: center; margin-top: 10px;">View All Records</a>
</div>
</body>
</html>
\end{lstlisting}

\lstset{language=PHP}
\begin{lstlisting}
<?php
// insert.php
$servername = "localhost";
$username = "root";
$password = "";
$database = "STUDENT";
$conn = new mysqli($servername, $username, $password, $database);
if ($conn->connect_error) {
    die("Connection failed: " . $conn->connect_error);
}
$stmt = $conn->prepare("INSERT INTO BCA (rollno, name, gender, phone) VALUES (?, ?, ?, ?)");
$stmt->bind_param("isss", $rollno, $name, $gender, $phone);
$rollno = $_POST['rollno'];
$name = $_POST['name'];
$gender = $_POST['gender'];
$phone = $_POST['phone'];
if ($stmt->execute()) {
    echo "Record inserted successfully!";
} else {
    echo "Error: " . $stmt->error;
}
$stmt->close();
$conn->close();
?>
\end{lstlisting}

\subsection{Deletion Code}
The deletion operation uses a PHP script (\texttt{delete.php}) to remove a record by roll number.

\lstset{language=PHP}
\begin{lstlisting}
<?php
// delete.php
$servername = "localhost";
$username = "root";
$password = "";
$database = "STUDENT";
$conn = new mysqli($servername, $username, $password, $database);
if ($conn->connect_error) {
    die("Connection failed: " . $conn->connect_error);
}
$rollno = $_GET['rollno'];
$stmt = $conn->prepare("DELETE FROM BCA WHERE rollno = ?");
$stmt->bind_param("i", $rollno);
if ($stmt->execute()) {
    header("Location: display.php");
} else {
    echo "Error deleting record: " . $stmt->error;
}
echo '<a href="display.php">Back to Records</a>';
$stmt->close();
$conn->close();
?>
\end{lstlisting}

\subsection{Update Code}
The update operation uses a PHP script (\texttt{update.php}) with an HTML form to modify existing records.

\lstset{language=PHP}
\begin{lstlisting}
<?php
// update.php
$servername = "localhost";
$username = "root";
$password = "";
$database = "STUDENT";
$conn = new mysqli($servername, $username, $password, $database);
if ($conn->connect_error) {
    die("Connection failed: " . $conn->connect_error);
}
$rollno = $_GET['rollno'];
if ($_SERVER['REQUEST_METHOD'] == 'POST') {
    $name = $_POST['name'];
    $gender = $_POST['gender'];
    $phone = $_POST['phone'];
    $stmt = $conn->prepare("UPDATE BCA SET name = ?, gender = ?, phone = ? WHERE rollno = ?");
    $stmt->bind_param("sssi", $name, $gender, $phone, $rollno);
    if ($stmt->execute()) {
        header("Location: display.php");
        exit();
    } else {
        echo "Error updating record: " . $stmt->error;
    }
    $stmt->close();
} else {
    $stmt = $conn->prepare("SELECT * FROM BCA WHERE rollno = ?");
    $stmt->bind_param("i", $rollno);
    $stmt->execute();
    $result = $stmt->get_result();
    $row = $result->fetch_assoc();
    $stmt->close();
}
?>
<!DOCTYPE html>
<html lang="en">
<head>
    <meta charset="UTF-8">
    <title>Update Record</title>
    <style>
        body {
            font-family: 'Segoe UI', Tahoma, Geneva, Verdana, sans-serif;
            margin: 20px;
        }
        h2 {
            text-align: center;
        }
        form {
            max-width: 400px;
            margin: 0 auto;
        }
        label {
            display: block;
            margin-bottom: 5px;
        }
        input, select {
            width: 100%;
            padding: 10px;
            margin-bottom: 10px;
            border: 1px solid #ccc;
            border-radius: 4px;
        }
        input[type="submit"] {
            background-color: #4CAF50;
            color: white;
            border: none;
            cursor: pointer;
        }
    </style>
</head>
<body>
    <h2>Update Record</h2>
    <form method="post">
        <label>Name:</label>
        <input type="text" name="name" value="<?= htmlspecialchars($row['name']) ?>" required>
        <label>Gender:</label>
        <select name="gender" required>
            <option value="Male" <?= $row['gender'] == 'Male' ? 'selected' : '' ?>>Male</option>
            <option value="Female" <?= $row['gender'] == 'Female' ? 'selected' : '' ?>>Female</option>
            <option value="Other" <?= $row['gender'] == 'Other' ? 'selected' : '' ?>>Other</option>
        </select>
        <label>Phone:</label>
        <input type="text" name="phone" value="<?= htmlspecialchars($row['phone']) ?>" required>
        <input type="submit" value="Update">
    </form>
</body>
</html>
<?php
$conn->close();
?>
\end{lstlisting}

\subsection{Display Code}
The display operation uses a PHP script (\texttt{display.php}) to show all records in a table.

\lstset{language=PHP}
\begin{lstlisting}
<?php
// display.php
error_reporting(E_ALL);
ini_set('display_errors', 1);
$servername = "localhost";
$username = "root";
$password = "";
$database = "STUDENT";
$conn = new mysqli($servername, $username, $password, $database);
if ($conn->connect_error) {
    die("Connection failed: " . $conn->connect_error);
}
echo "Connected successfully<br>";
$query = "SELECT * FROM BCA";
$data = mysqli_query($conn, $query);
if (!$data) {
    die("Query failed: " . mysqli_error($conn));
}
$total = mysqli_num_rows($data);
if ($total != 0) {
    echo "<table border='1' cellspacing='0' cellpadding='10'>";
    echo "<tr><th>rollno</th><th>name</th><th>gender</th><th>phone</th><th>Update</th><th>Delete</th></tr>";
    while ($result = mysqli_fetch_assoc($data)) {
        echo "<tr>";
        echo "<td>" . $result['rollno'] . "</td>";
        echo "<td>" . htmlspecialchars($result['name']) . "</td>";
        echo "<td>" . $result['gender'] . "</td>";
        echo "<td>" . $result['phone'] . "</td>";
        echo "<td><a href='update.php?rollno=" . $result['rollno'] . "'>Update</a></td>";
        echo "<td><a href='delete.php?rollno=" . $result['rollno'] . "' onclick='return confirm(\"Are you sure?\");'>Delete</a></td>";
        echo "</tr>";
    }
    echo "</table>";
} else {
    echo "No records found.";
}
$conn->close();
?>
<a href="form.php" style="display: block; text-align: center; margin-top: 10px;">Back to Form</a>
\end{lstlisting}


\subsection{CRUD Output}
The rendered output of the CRUD operations (e.g., form or display table) is shown below.

\begin{figure}[H]
    \centering
    \includegraphics[width=0.8\textwidth]{5_screenshot.png}
    \caption{Form Output}
\end{figure}

\begin{figure}[H]
    \centering
    \includegraphics[width=0.8\textwidth]{6_screenshot.png}
    \caption{CRUD Operations Output(display.php)}
\end{figure}

\begin{figure}[H]
    \centering
    \includegraphics[width=0.95\textwidth, height=0.6\textheight]{7_screenshot.png}
    \caption{CRUD Operations Output(database)}
\end{figure}

\section{Conclusion}
This lab report presents a complete student registration system with HTML, CSS, PHP, MySQL, and XML, with all CRUD operations fully implemented, output images included, and formatted consistently.

\end{document}



\section{Conclusion}
This lab report presents a complete student registration system with HTML, CSS, PHP, MySQL, and XML, with all CRUD operations fully implemented, output images included, and formatted consistently.

\end{document}
